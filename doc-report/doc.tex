\documentclass[lang=cn]{elegantpaper}
\usepackage{listings}
\usepackage{hyperref}
\title{编译原理 -- 实验 1 - 词法分析}
\author{61519322 杨哲睿}
\date{\today}

\begin{document}

\maketitle

\tableofcontents


% 提交清单:
% - Mini语言的词法规则描述(正规式或正规文法)
% - 正规式或正规文法->NFA->DFA->minimized DFA的手工过程(拍照整理为word或pdf文档)
% - 词法分析程序
% - 测试用例文件和测试结果文件
% - 实验报告

% 选择语言:pascal
% 源语言: rust-lang

\section{引言}

\subsection{实验要求}

\begin{quote}
    选择一个你熟悉的程序设计语言,找到它的规范(referrence or standard)。在规范中找到其词法的BNF或正规式描述。
\end{quote}

\begin{remark}
    选择的程序设计语言为:\lstinline|pascal|(源语言),具体规范参考为 \hyperref{https://github.com/bonzini/flex/blob/master/examples/manual/pascal.lex}{}{}{Github-pascal词法的lex描述}。
\end{remark}

\begin{quotation}
    选择该语言的一个子集(能够构成一个mini的语言,该语言至少能够进行函数调用、控制流语句(分支或循环)、简单的运算和赋值操作。)给出该mini语言的词法的正规文法或正规式。
    
\end{quotation}

在这里,选择的mini语言关键字为:

\begin{lstlisting}[language=pascal]
and
begin
div
do
else
end
for
function
if
in
nil
not
of
procedure
program
repeat
set
then
to
until
var
while
\end{lstlisting}
与标识符一同识别。

relop 如下:

\begin{lstlisting}[language=pascal]
<= >= < > <> =
:=
..
+ - * /
\end{lstlisting}

界符 sep 如下:
\begin{lstlisting}[language=pascal]
[ ] ( )
, ; .
\end{lstlisting}

其他需要的基本词法单元如下:
\begin{enumerate}
    \item 数
    \item 字符串
\end{enumerate}

下面给出具体的词法规则描述。

\section{词法规则描述、Minimized-DFA 简化过程}

\section{词法分析程序}

\subsection{简述}

\subsection{测试用例和测试结果}

\section{总结}

%%%%%%%%%%%%%%%%%%%%%%%%%%%%%%%%%%%%%%%%%%%%%%%%%%%%%%%%%%%
%%%%%%%%%%%%%%%%%%%%%%%%%% Biblo %%%%%%%%%%%%%%%%%%%%%%%%%%

% \begin{thebibliography}{99}  

    
% \end{thebibliography}

%%%%%%%%%%%%%%%%%%%%%%%%%%%%%%%%%%%%%%%%%%%%%%%%%%%%%%%%%%%
\end{document}