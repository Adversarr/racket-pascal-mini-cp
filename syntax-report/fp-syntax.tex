\documentclass[lang=cn]{ctexart}
\usepackage{hyperref}
\usepackage{graphicx}
\usepackage{listings}
\usepackage{geometry}
\usepackage{color}

\usepackage{amsmath}

\newtheorem{remark}{注}

\geometry{
    a4paper,
    left=2cm,right=2cm,
    top=2.5cm,bottom=2.5cm
}

\lstset{
	language=lisp,
	basicstyle=\tt,
	numbers=left, numberstyle=\tiny,
	xleftmargin=5em,
	xrightmargin=5em
}

\title{
    LR(1) 语法分析
    \begin{flushright}\normalsize
        \textit{-- Based on Racket (A Dialect of Lisp)}
    \end{flushright}
}

\author{61519322 杨哲睿}

\begin{document}
\maketitle

\tableofcontents
\newpage

\section{实验要求}

\begin{enumerate}
    \item 手工画出LR(1)项目集族及状态图
    \item 构造LR(1)分析表
    \item 在上述过程中,对二义文法进行必要的处理
    \subitem 表达优先级和结合律
    \subitem 按照优先级和结合律对预测符及状态转换关系进行必要的取舍
    \item 按照LR(1)分析表写出语法分析程序
\end{enumerate}

当然,这只是一个最简单的要求。但考虑到LR(1)分析器的手工构造难度,该实验使用{\heiti 程序推导}的方式来构造LR(1)项目集族和状态转换图。并且在实验的具体实现上有所变化。

\section{设计思路}

根据 \hyperref{https://htdp.org/2021-11-15/Book/part_preface.html#%28part._sec~3asystematic-design%29}{website}{}{htdp} 中的详细介绍,我们可以将设计一个程序分为以下六个步骤:

\begin{enumerate}
    \item From Problem to Data Definitions
    \item Signature, Purpose Statement, Header
    \item Functional Examples
    \item Function Template
    \item Function Definitions
    \item Testing
\end{enumerate}

报告也将从这六个方面展开,对于该语法分析程序的制作进行分析。

\subsection{From Problem to Data Definitions}

\begin{quotation}\textit{
    Identify the information that must be represented and how it is represented in the chosen programming language. Formulate data definitions and illustrate them with examples.}
\end{quotation}

总的来说,我们LR分析程序:
\begin{itemize}
	\item 输入为:Token 的流 (具体实现为\textit{生成器generator})
	\item 输出为:对应的一颗语法树
\end{itemize}

一个LR(1)语法分析程序所涉及的数据结构如下:

\begin{enumerate}
    \item \lstinline|syntax-item| 文法符号 -- 指上下文无关文法中的\textit{非终结符}或\textit{终结符}单元,拥有名称 \lstinline|id|,优先级 \lstinline|proority|、匹配器(函数)\lstinline|matcher|
    \item \lstinline|matched-item| 匹配的文法符号 -- 指输出时匹配后的\textit{终结符}单元
    \item \lstinline|production| 产生式 -- 上下文无关文法的\textit{一个}产生式,分为左部和右部两部分
    \item \lstinline|syntax-tree-node| 语法树的一个节点
    \item \lstinline|LRItem| LR(1) 项 -- 分为四个部分:\textit{产生式的左部、产生式右部在$\cdot$之前的部分、产生式右部在$\cdot$之后的部分、展望(Look-Ahead)符号}
\end{enumerate}

对该数据结构的定义如下:

\begin{lstlisting}[caption={types.rkt}]
(struct syntax-item
 (id priority matcher) #:transparent)
(struct matched-item
 (stx content))
(struct production
 (left right) #:transparent)
(struct syntax-tree-node
 (head children) #:transparent)
\end{lstlisting}

当然这些只是基础的定义,为了完成LR(1)程序,我们还需要定义:

\begin{enumerate}
	\item 一个 LR(1) 项目集就是一个 \lstinline|LRItem| 的\textit{列表}
	\item 一个上下文无关文法包含:\textit{终结符号}、\textit{非终结符号}、\textit{开始符号}、\textit{产生式}
	\item 对于一个增广的上下文无关文法,还需要一个\textit{增广文法的新开始符号}及其对应的\textit{开始符号产生式}
\end{enumerate}

\subsection{Signature, Purpose Statement, Header and Functional Examples}

\begin{quotation}
	\textit{State what kind of data the desired function consumes and produces. Formulate a concise answer to the question what the function computes. Define a stub that lives up to the signature.}
\end{quotation}

根据这些类型的定义,我们足以定义出整个LR(1)语法分析程序的所有函数,如下:

\paragraph{closure}
\begin{description}
	\item[输入] LR(1) 项目集 $I$
	\item[输出] 项目集的闭包 $\mathrm{CLOSURE}(I)$
\end{description}

\begin{remark}
	使用柯里化,这里不需要输入产生式表。
\end{remark}

\paragraph{get-closure-function}
\begin{description}
	\item[输入] 产生式的列表
	\item[输出] $\mathrm{CLOSURE}$ 函数
\end{description}

\begin{remark}
	同理下面几个函数也使用柯里化,降低复杂度。
\end{remark}

\paragraph{get-look-ahead}
\begin{description}
	\item[输入] 产生式列表
	\item[输出] $look-ahead$函数
\end{description}

\paragraph{look-ahead}
\begin{description}
	\item[输入] LRItem
	\item[输出] 当前LRItem在获取到下一个非终结符后的部分的First集
\end{description}
\begin{remark}
	若有 $\mathrm{LRItem}= [A\rightarrow \alpha \cdot B \beta, a]$ 则$look-ahead(\mathrm{LRItem}) = \textsc{FIRST}(\beta)$。这里不需要原来的look-ahead是因为在后续计算中加入了look-ahead的考量。
\end{remark}

\paragraph{go}
\begin{description}
	\item[输入] LR(1)项目集闭包、syntax-item
	\item[输出] 闭包在获取到该syntax-item后产生的新项目集(非闭包)
\end{description}

\paragraph{look-ahead-and-reduce}
\begin{description}
	\item[输入] LR(1)项目集闭包、(读头下的非终结符对应的)syntax-item
	\item[输出] 闭包中可以执行规约的项目列表
\end{description}

\begin{remark}
	这是我实现的方法和书上方法主要的不同:在龙书中,LR自动机的一个项目实则对应了一个项目集(闭包),所以我们无需对于状态进行编号,而只需要通过闭包的转换即可实现等价的自动机行为。这里输出的的可规约的项即为在预测分析表中填上的 $r$,而使用的产生式可以直接通过该规约项所对应的产生式计算得到。
\end{remark}

\begin{remark}
	实际上,我们不需要返回一个列表,因为LR(1)不产生规约-规约冲突。
\end{remark}

\paragraph{try-to-shift-in}
\begin{description}
	\item[输入] LR(1)项目集闭包、(读头下的非终结符对应的)syntax-item
	\item[输出] 闭包中可以执行移进的项目
\end{description}




\end{document}