\documentclass[lang=cn]{ctexart}
\usepackage{hyperref}
\usepackage{graphicx}
\usepackage{listings}
\usepackage{geometry}
\usepackage{color}

\usepackage{amsmath}

\newtheorem{remark}{注}

\geometry{
    a4paper,
    left=2cm,right=2cm,
    top=2.5cm,bottom=2.5cm
}

\lstset{
	language=lisp,
	basicstyle=\tt,
	numbers=left, numberstyle=\tiny,
	xleftmargin=5em,
	xrightmargin=5em
}

\title{
    LR(1) 语法分析
    \begin{flushright}\normalsize
        \textit{-- Based on Racket (A Dialect of Lisp)}
    \end{flushright}
}

\author{61519322 杨哲睿}

\begin{document}
\maketitle

\tableofcontents
\newpage

\section{实验要求}

\begin{enumerate}
    \item 手工画出LR(1)项目集族及状态图
    \item 构造LR(1)分析表
    \item 在上述过程中,对二义文法进行必要的处理
    \subitem 表达优先级和结合律
    \subitem 按照优先级和结合律对预测符及状态转换关系进行必要的取舍
    \item 按照LR(1)分析表写出语法分析程序
\end{enumerate}

当然,这只是一个最简单的要求。但考虑到LR(1)分析器的手工构造难度,该实验使用{\heiti 程序推导}的方式来构造LR(1)项目集族和状态转换图。并且在实验的具体实现上有所变化。

\section{设计思路}

根据 \hyperref{https://htdp.org/2021-11-15/Book/part_preface.html#%28part._sec~3asystematic-design%29}{website}{}{htdp} 中的详细介绍,我们可以将设计一个程序分为以下六个步骤:

\begin{enumerate}
    \item From Problem to Data Definitions
    \item Signature, Purpose Statement, Header
    \item Functional Examples
    \item Function Template
    \item Function Definitions
    \item Testing
\end{enumerate}

报告也将从这六个方面展开,对于该语法分析程序的制作进行分析。

\subsection{From Problem to Data Definitions}

\begin{quotation}\textit{
    Identify the information that must be represented and how it is represented in the chosen programming language. Formulate data definitions and illustrate them with examples.}
\end{quotation}

总的来说,我们LR分析程序:
\begin{itemize}
	\item 输入为:Token 的流 (具体实现为\textit{生成器generator})
	\item 输出为:对应的一颗语法树
\end{itemize}

一个LR(1)语法分析程序所涉及的数据结构如下:

\begin{enumerate}
    \item \lstinline|syntax-item| 文法符号 -- 指上下文无关文法中的\textit{非终结符}或\textit{终结符}单元,拥有名称 \lstinline|id|,优先级 \lstinline|proority|、匹配器(函数)\lstinline|matcher|
    \item \lstinline|matched-item| 匹配的文法符号 -- 指输出时匹配后的\textit{终结符}单元
    \item \lstinline|production| 产生式 -- 上下文无关文法的\textit{一个}产生式,分为左部和右部两部分
    \item \lstinline|syntax-tree-node| 语法树的一个节点
    \item \lstinline|LRItem| LR(1) 项 -- 分为四个部分:\textit{产生式的左部、产生式右部在$\cdot$之前的部分、产生式右部在$\cdot$之后的部分、展望(Look-Ahead)符号}
\end{enumerate}

对该数据结构的定义如下:

\begin{lstlisting}[caption={types.rkt}]
(struct syntax-item
 (id priority matcher) #:transparent)
(struct matched-item
 (stx content))
(struct production
 (left right) #:transparent)
(struct syntax-tree-node
 (head children) #:transparent)
\end{lstlisting}

当然这些只是基础的定义,为了完成LR(1)程序,我们还需要定义:

\begin{enumerate}
	\item 一个 LR(1) 项目集就是一个 \lstinline|LRItem| 的\textit{列表}
	\item 一个上下文无关文法包含:\textit{终结符号}、\textit{非终结符号}、\textit{开始符号}、\textit{产生式}
	\item 对于一个增广的上下文无关文法,还需要一个\textit{增广文法的新开始符号}及其对应的\textit{开始符号产生式}
\end{enumerate}

\subsection{Signature, Purpose Statement, Header and Functional Examples}

\begin{quotation}
	\textit{State what kind of data the desired function consumes and produces. Formulate a concise answer to the question what the function computes. Define a stub that lives up to the signature.}
\end{quotation}

根据这些类型的定义,我们足以定义出整个LR(1)语法分析程序的所有函数,如下:

\paragraph{closure}
\begin{description}
	\item[输入] LR(1) 项目集 $I$
	\item[输出] 项目集的闭包 $\mathrm{CLOSURE}(I)$
\end{description}

\begin{remark}
	使用柯里化,这里不需要输入产生式表。
\end{remark}

\paragraph{get-closure-function}
\begin{description}
	\item[输入] 产生式的列表
	\item[输出] $\mathrm{CLOSURE}$ 函数
\end{description}

\begin{remark}
	同理下面几个函数也使用柯里化,降低复杂度。
\end{remark}

\paragraph{get-look-ahead}
\begin{description}
	\item[输入] 产生式列表
	\item[输出] $look-ahead$函数
\end{description}

\paragraph{look-ahead}
\begin{description}
	\item[输入] LRItem
	\item[输出] 当前LRItem在获取到下一个非终结符后的部分的First集
\end{description}
\begin{remark}
	若有 $\mathrm{LRItem}= [A\rightarrow \alpha \cdot B \beta, a]$ 则$look-ahead(\mathrm{LRItem}) = \textsc{FIRST}(\beta)$。这里不需要原来的look-ahead是因为在后续计算中加入了look-ahead的考量。
\end{remark}

\paragraph{go}
\begin{description}
	\item[输入] LR(1)项目集闭包、syntax-item
	\item[输出] 闭包在获取到该syntax-item后产生的新项目集(非闭包)
\end{description}

\paragraph{look-ahead-and-reduce}
\begin{description}
	\item[输入] LR(1)项目集闭包、(读头下的非终结符对应的)syntax-item
	\item[输出] 闭包中可以执行规约的项目列表
\end{description}

\begin{remark}
	这是我实现的方法和书上方法主要的不同:在龙书中,LR自动机的一个项目实则对应了一个项目集(闭包),所以我们无需对于状态进行编号,而只需要通过闭包的转换即可实现等价的自动机行为。这里输出的的可规约的项即为在预测分析表中填上的 $r$,而使用的产生式可以直接通过该规约项所对应的产生式计算得到。
\end{remark}

\begin{remark}
	实际上,我们不需要返回一个列表,因为LR(1)不产生规约-规约冲突。
\end{remark}

\paragraph{try-to-shift-in}
\begin{description}
	\item[输入] LR(1)项目集闭包、(读头下的非终结符对应的)syntax-item
	\item[输出] 闭包中可以执行移进的项目列表
\end{description}


\paragraph{try-to-shift-in}
\begin{description}
	\item[输入] LR(1)项目集闭包、(读头下的非终结符对应的)syntax-item
	\item[输出] 闭包中可以执行移进的项目列表
\end{description}



\paragraph{build-lr1-automata}
\begin{description}
	\item[输入] 产生式列表、增广文法的开始符号的产生式、EOF符号
	\item[输出] 可以在一个token流上执行的自动机
\end{description}

\begin{remark}
	对应于龙书中的LR(1)算法主体,也就是所谓的LR(1)分析的主控。
\end{remark}

\subsubsection{LR自动机}

从实现上说,实际上这不是一个自动机,而是一个先柯里化后的递归函数!

假设我们有一个词法单元(Token,记为$T$)组成的列表 $t_1, t_2, \cdots, t_n, \#$,LR(1)语法分析的输出应该是一颗语法树。简单说即为:

$$
f_1:T\times T\times\cdots \times T \rightarrow \{\mathrm{Syntax~Tree}\}
$$

使用柯里化后:

$$
\begin{aligned}
	f_1:&t_1\mapto f_2\\
	f_2:&t_2\mapto f_3\\
	&\cdots\\
	f_n:&t_n \mapto f_{n+1}\\
	f_{n+1}:&\# \mapto \mathrm{Syntax~Tree}
\end{aligned}
$$

可以看出,其输入都是相同的,而提供给函数的实际上是一个流对象(生成器),所以考虑最后将其转换为递归函数实现。在实验1中,已经说明了:状态机的实质就是从函数和输入到另一个函数的映射(状态转换函数)。因此,在语法分析实现过程中,将LR自动机实现为这样的函数。但实现中不难发现一个问题,由于我们在编码时无法确定状态总数,状态转换函数都定义为匿名函数。也就是说,这个递归函数跳转时跳转的对象是没有事先绑定的,显然,这是一个Y-组合子问题。

当然,我们也可以做出一点改变,用一个很不函数式的方法来实现类似于一个Y-组合子的方法,我们将每一个状态及其对应的状态转换函数,写入一个哈希表中,在进行状态跳转的同时在哈希表里面查询相应的表项执行,来避免Y-组合子中无法进行匿名函数变量绑定的问题。

\subsection{Function Template}
\begin{quotation}
	\textit{Translate the data definitions into an outline of the function.}
\end{quotation}

在此,我仅仅给出 \lstinline|build-lr1-automata| 的定义过程。

\subsection{LR(1)自动机构造基础}

给定产生式列表、增广文法的开始符号产生式,我们可以计算如下内容:

\begin{enumerate}
	\item $\mathrm{CLOSURE}$ 函数(Currying)
	\item $I_0$ 即自动机的初始状态所对应的项集闭包
	\item $\mathrm{GOTO}$ 函数
\end{enumerate}

\begin{remark}
	在这里的 $\mathrm{GOTO}$ 函数和先前定义的 $\mathrm{GO}$ 函数是不同的,GO函数仅仅计算出从闭包 $I_1$ 通过输入文法符号 $A$ 得到的下一个闭包 $I_2$,但在这里,$I_2$ 是通过$I_1$推导的,虽然本质上重复两次计算 $\mathrm{GO}(I_1, A)$ 所得到的 $I_2$ 是相同的(函数是单值映射),但在计算机内部,由于通过了两次推导,用面向对象的说法即是两个截然不同的对象。
\end{remark}

注意,上面实现的 $\mathrm{GOTO}$ 函数是可以接受终结符号的!在报告实现的语法分析器中,由于将自动机的状态和LR(1)项目集构建了一一对应关系,即自动机运行时能确定自身状态对应的LR(1)项目集,可以通过该项目集和输入的符号和$\mathrm{GOTO}$函数来确定执行移进或者规约。换言之,,不需要用 $\mathrm{ACTION}$ 函数即可实现该自动机。实现如下:考虑原有的移进和规约的执行逻辑,
	
在LR(1)分析表中填入移进,当下面两项中有且仅有一项成立:

\begin{enumerate}
	\item $\mathrm{GO}(I_1, a) = I_2$ 且 $a$ 是终结符号,那么在 $\mathrm{ACTION}$ 中填入移进并且转状态 $I_2$(项目集对应的状态)
	\item $\mathrm{GO}(I_1, A) = I_2$ 且 $A$ 是非终结符号,那么在 $\mathrm{GOTO}$ 中填入状态 $I_2$
\end{enumerate}

重点关注非终结符号情况:在LR自动机执行的过程中,\textit{必定是在规约得到了一个非终结符号后执行}$\mathrm{GOTO}$\textbf{操作}。对于这类情况我们可以简单的执行如下三个内容:把\textit{读头}下的内容转换为该非终结符号,并重新执行移进所对应的逻辑即可。

\begin{remark}
	这样的设计是合理的,对于一般的读头下的token,我们在产生时先存入
\end{remark}


\end{document}